\documentclass{article}

\usepackage[ngerman]{babel}
\usepackage[utf8]{inputenc}
\usepackage[T1]{fontenc}
\usepackage{hyperref}
\usepackage{csquotes}

\usepackage[
    backend=biber,
    style=apa,
    sortlocale=de_DE,
    natbib=true,
    url=false,
    doi=false,
    sortcites=true,
    sorting=nyt,
    isbn=false,
    hyperref=true,
    backref=false,
    giveninits=false,
    eprint=false]{biblatex}
\addbibresource{../references/bibliography.bib}

\title{Review des Papers "KI in der Medizin" von Leonard Gaa\dots}
\author{Loïc Schneider}
\date{\today}

\begin{document} 
\pagestyle{empty}
\begin{titlepage} 
\maketitle
\thispagestyle{empty} 
\vspace{5em}

\abstract{
\noindent
Dies ist ein Review der Arbeit \textquote{KI in der Medizin} von Leonard Gaa in der Fassung vom 31. Mai 2024. Darin soll die Anwendung von KI in der Medizin, sowie die ethischen Bedenken, die dies mit sich bringt, thematisiert werden.
}
\end{titlepage}
\newpage

\section{Kritik}
Die Arbeit behandelt das Thema rund um KI in der Medizin sehr gut, es wurde auf verschiedene Aspekte eingegangen und stehts mehrere Ansichten aufgezeigt.
Es wird ein guter einstieg in dieses komplexe aber doch sehr aktuelle Thema geboten - inhaltlich informativ.
Die Arbeit sensibilisiert einen mehr für das Thema und kann Vertrauen schaffen.

Die Beschreibung zum lernen der KI gibt einen guten Einstieg ins Thema, allerdings ist sie noch ausbaufähig. Der Vorgang des maschinellen Lernens und \textquote{Deep Learning} hätten genauer erklärt werden können, um den Leser abzuholen.

Durch konkrete Beispiele wie die Bildauswertung und die Smartwatches wurde erzähltes durch logische Situationen untermahlt. Man kann sich dadurch also etwas unter den theoretischen Aspekten vorstellen.

Durch die inteplementierte Grafik konnte das erzählte veranschaulicht werden, der Einstatz von Grafiken ist ein gutes Stilmittel und macht hier auch Sinn.

Auch die ehtischen Fragen rund um das Thema wurden erwähnt, hätten teilweise aber noch ein wenig mehr ausgearbeitet sein dürfen.
Diese sind wie man auch aus der Arbeit erkennen kann sehr relevant und sollten stehts präsent sein, daher wäre in diesem Bereich ein wenig mehr tiefe wünschenswert.
(Konsequenzen, weitere Aspekte, mögliche Vorgehensweise um dagegen anzukämpfen...)

\section{Verbesserungsvorschläge}
Konkret würde ich zuerst einmal den Text bereinigen und vereinzelte Satzstrukturen anpassen um das Lesesn an gewissen Stellen angehmer zu gestallten, sowie die paar Sprachlichen Fehler korrigieren.

Manche bereiche wie den Input zum Lernen bei künstlichen Intelligenzen sollte man noch weiter ausführen. Sowie auch die ethischen Aspekte noch ein bischen mehr gewichten.

Eine konkrete Fragestellung an den Anfang der Arbeit stellen würde die Struktur noch klarer gestallen.

\section{Schlussowrt}
Insgesammt ist die Arbeit logisch aufgebaut und sinnvoll strukturiert. Es wurden recherchen zum Oberthema durchegführt und die Vor- und Nachteile sind klar dargestellt. 
Die Grafik wurde gut implementiert und die Quellen sind alle sehr aktuell, was eine gute Grundlage ist und in einem so aktuellen Thema auch wichtig, da stehts neue erkenntnisse geschlossen werden.
Stellenweise sind noch Fehler auszubessern und eine Fragestellung, die in der Arbeit beantwortet wird wäre Stilistisch schön gewesen.

\printbibliography

\end{document}
