\chapter{Fazit}
\label{chap:Fazit}

Die Integration von Künstlicher Intelligenz in die Medizin bietet sowohl immense Chancen als auch erhebliche Risiken. 

KI kann die Medizin effizienter, präziser und patientenorientierter gestalten, indem sie Ärzte entlastet und durch ihre Fähigkeit die Diagnose und Behandlung von Krankheiten verbessert. 
Zudem kann KI die Forschung nach neuen Medikamenten und Behandlungsmethoden unterstützen und durch Fernüberwachung oder virtuelle Assistenten Patienten auch außerhalb von medizinischen Einrichtungen helfen.

Dennoch sind diese Vorteile nicht ohne Herausforderungen. Ethische Bedenken, Datenschutzfragen und die Zuverlässigkeit der KI-Systeme stellen bedeutende Risiken dar. 
Die Skepsis der Patienten gegenüber der Sicherheit und Funktionalität der KI kann zu einer Ablehnung von Behandlungen führen, was wiederum physische und psychische Folgen haben kann. 
Der Umgang mit sensiblen Patientendaten erfordert strikte Datenschutzmaßnahmen.
Ethische Überlegungen betreffen die Frage, wie viel Verantwortung an Maschinen abgegeben werden sollte, die letztlich auf algorithmischen Entscheidungen basieren und keine moralische Urteilsfähigkeit besitzen. 
Rechtliche Fragen zur Haftung im Falle von Fehlern durch KI-Systeme müssen geklärt werden, und es ist wichtig, dass Entscheidungen, die von KI getroffen werden, für Ärzte und Patienten nachvollziehbar sind.

\vspace{5mm} \noindent
Insgesamt muss der Einsatz von KI in der Medizin sorgfältig geplant und überwacht werden, um sicherzustellen, dass die Vorteile maximiert und die Risiken minimiert werden. 
Datenschutz, ethische Prinzipien und rechtliche Rahmenbedingungen müssen dabei stets berücksichtigt werden, um den sicheren und verantwortungsvollen Einsatz von KI in der Gesundheitsversorgung zu gewährleisten.