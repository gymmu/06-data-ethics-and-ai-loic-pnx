\chapter{KI in der Medizin}
\label{chap:ai-medicine}

Ein grosses potenzielles und auch teilweise schon exisiterendes Anwendungsgebiet für Künstliche Intelligenzen liegt in der Medizin.
Dort kann diese vielseitig, beispielsweise für Diagnosen oder eine Ersteinschätzung von Patienten eingesetzt werden.

Dabei entstehen grosse Chancen wie aber auch Risiken.

\section{Chancen}
Dem vermehrten Einsatz von KI in der Medizin stehen viele Chancen gegenüber.
Ein Problem in der Medizin ist, dass sich Ärzte zu wenig Zeit nehmen können um wirklich auf die Patienten einzugehen und eine Verbindung aufzuabauen.
Wenn KI gewisse Aufgaben übernehmen kann, würde das die Ärzte entlasten und sie könnten sich wieder verstärkt auf die Patienten einlassen, was den Arbeitsaltag der Ärzte erleichtert so wie dem Patienten ein besseeres Gefühl zu vermitteln.

Ein erster Schritt zur implementierung von KI wäre ein Programm in dem der Patient sein Krankheitsbild beschreiben kann um Indormationen zu möglichen Ursachen zu bekommen. Anhand von diesen Informationen könnte man dann die Dringlichkeit eines Arztbesuches abwägen.

Ein logischer und auch schon verwendeter weiterer Schritt ist die Krankheitsdiagnose anhand von Patientendaten. 
Der Vorteil gegenüber einer menschlichen Diagnose ist dabei, dass KI viel besser aus verschiedensten Daten Zusammenhänge erkennen kann. 
Dadurch werden menschliche Fehler reduziert und das Erkennen von Krankheiten wird in Fällen möglich, bei denen einem Arzt eventuell keine Unüblichkeiten aufgefallen wären.

Auch in der Forschung nach neuen Medikamenten und Praktiken in der Medizin, kann KI behilflich sein. In grossen Datenmengen findet KI potenzielle neue Wirkstoffe.
Ausserdem können Ergebnisse von Studien sowie effizient ausgewertet werden.

Bei körperlichen Beschwerden und auch chronischen Krankheiten hat man nicht in jeder Situation eine Fachperson zur Stelle. In diesen Momenten könnte eine KI basierte Fernüberwachung oder eine Art Virtueller Assistent Abhilfe schaffen.

\vspace{3mm} \noindent
Durch den Einsatz von KI in diesen Bereichen kann die Medizin effizienter, präziser und patientenorientierter gestaltet werden. 
Gleichzeitig müssen jedoch ethische Aspekte und der Datenschutz berücksichtigt werden, um den sicheren und verantwortungsvollen Einsatz von KI in der Gesundheitsversorgung zu gewährleisten.

\section{Risiken}
Mit den vielversprechenden Chancen von künstlicher Intelligenz in Branchen wie der Medizin kommen aber auch einige Risiken. Bevor man ein KI System in den Alltag implementieren kann, solltem an seine Risiken abwägen.
Bevor ein System mit echten Daten in den Alltag implementiert wird, sollte es die Risiken auf eine kleinst mögliche Zahl reduizeren und ein sauber ausgearbeitetes konzept haben.


\subsection{Skepsis und Datenschutz}
Ein Risiko für welches das System nicht direkt etwas dafür kann, ist die Skepsis von Menschen, gegenüber der Funktionalität oder Sicherheit. 
Dadurch können sich Menschen weigern gewisse Behandlungen anzunehmen. Das kann körperliche Ausqirkungen auf den Patienten haben, aber auch einen psychischen Graben zwischen Kranken und dem Gesundheitswesen schaffen.

Ein Grund für Skepsis kann der Datenschutz sein. Beim training der KI aber auch während sie ihren Aufgaben in der Medizin nachkommt, werden grosse Mengen an Daten benötigt respektive gesammelt. 
Darunter befinden sich Daten über Krankheiten von Patienten und etlichen Körperwerte, die niemanden ausser den Arzt und einen selbst etwas angehen und auch nicht wünschenswert ist, dass irgend ein Fremder an diese gelangt.
Gerade Patientendaten von Prominenten sind sehr wertvoll und steigern das Risiko von Cyberangriffen. 

\subsection{Ethik}
Ein weiteres Risiko sind Fehler der KI. Daher muss man sich die Frage stellen, wie viel Verantwortung man der Maschine abgibt.
Inwiefern ist es überhaupt ethisch Menschenleben von System, die durch Strom an, Strom aus funktioniern, abhängig zu machen?

Im prinzip werden Menschenleben durch KI in der Medizin auf blosse Datensätze reduziert. Wo bleibt da die Menschenwürde?
Ein moralisches bzw. ethisches Problem liegt auch darin, dass KI genau diese beiden Dinge momentan nur spärlich besitzt. 
Durch Daten lassen sich zwar rational  die besten Entscheidungen finden, allerdings ist eine komplett rationale Entscheidung teilweise nicht die optimale.

\subsection{Rechtliches}
Weiter muss man sich überlegen, wer genau Verantwortlich ist, wenn künstliche Intelligenz einen Fehler macht und inwiefern wer rechtliche Konsequenzen tragen kann.


\vspace{1cm}
Datenschutz
Entscheidungen
Haftung
Ethik
