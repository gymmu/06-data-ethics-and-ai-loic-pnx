\chapter{KI in der Medizin}
\label{chap:ai-medicine}

Ein grosses potenzielles und auch teilweise schon exisiterendes Anwendungsgebiet für Künstliche Intelligenzen liegt in der Medizin.
Dort kann diese vielseitig, beispielsweise für Diagnosen oder eine Ersteinschätzung von Patienten eingesetzt werden.

Dabei entstehen grosse Chancen wie aber auch Risiken.

\section{Chancen}
Dem vermehrten Einsatz von KI in der Medizin stehen viele Chancen gegenüber.
Ein Problem in der Medizin ist, dass sich Ärtze zu wenig Zeit nehmen können um wirklich auf die Patienten einzugehen und eine Verbindung aufzuabauen.
Wenn KI gewisse Aufgaben übernehmen kann, würde das die Ärtzte entlasten und sie könnten sich wieder verstärkt auf die Patienten einlassen, was den Arbeitsaltag der Ärtze erleichtert so wie dem Patienten ein besseeres Gefühl zu vermitteln.

Ein erster Schritt zur implementierung von KI wäre ein Programm in dem der Patient sein Krankheitsbild beschreiben kann um Indormationen zu möglichen Ursachen zu bekommen. Anhand von diesen Informationen könnte man dann die Dringlichkeit eines Artztbesuches abwägen.

Ein logischer und auch schon verwendeter weiterer Schritt ist die Krankheitsdiagnose anhand von Patientendaten. 
Der Vorteil gegenüber einer menschlichen Diagnose ist dabei, dass KI viel besser ist aus verschiedensten Daten Zusammenhänge zu erkennen. 
Dadurch werden menschliche Fehler reduziert und das Erkennen von Krankheiten wird in Fällen möglich, bei denen einem Artz eventuell keine Unüblichkeiten aufgefallen wären.

Auch in der Forschung nach neuen Medikamenten und Praktiken in der Medizin, kann KI behilflich sein. In grossen Datenmengen findet KI potenzielle neue Wirkstoffe.
Ausserdem können Ergebnisse von Studien sowie effizient ausgewertet werden.

Bei körperlichen Beschwerden und auch chronischen Krankheiten hat man nicht in jeder Situation eine Fachperson zur Stelle. In diesen Momenten könnte eine KI basierte Fernüberwachung oder eine Art Virtueller Assistent Abhilfe schaffen.

\vspace{3mm} \noindent
Durch den Einsatz von KI in diesen Bereichen kann die Medizin effizienter, präziser und patientenorientierter gestaltet werden. 
Gleichzeitig müssen jedoch ethische Aspekte und der Datenschutz berücksichtigt werden, um den sicheren und verantwortungsvollen Einsatz von KI in der Gesundheitsversorgung zu gewährleisten.

\section{Risiken}
Skepsis
Datenschutz
Entscheidungen
