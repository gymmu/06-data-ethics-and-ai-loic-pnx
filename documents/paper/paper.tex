\documentclass{report}

\usepackage[ngerman]{babel}
\usepackage[utf8]{inputenc}
\usepackage[T1]{fontenc}
\usepackage{hyperref}
\usepackage{csquotes}
\usepackage[a4paper]{geometry}

\usepackage[
    backend=biber,
    style=apa,
    sortlocale=de_DE,
    natbib=true,
    url=false,
    doi=false,
    sortcites=true,
    sorting=nyt,
    isbn=false,
    hyperref=true,
    backref=false,
    giveninits=false,
    eprint=false]{biblatex}
\addbibresource{../references/bibliography.bib}


\title{Künstliche Intelligenz und deren Auswirkung auf die Arbeitswelt}
\author{Loïc Schneider}
\date{\today}


\begin{document}

\begin{titlepage}
    \makeatletter % Das hier brauchen wir damit wir spezielle Befehle wie \@author verwenden können.
	\begin{center}
		{\scshape Gymnasium Muttenz} \vspace{0.5cm}

		 Informatik 2023/2024\vspace{5.5cm}

		{\huge\bfseries \@title}

		\vspace{2cm}

		{\Large\itshape \@author}

        \vspace{2cm}

        Version vom: \@date
	\end{center}
    
    \makeatother % Wir müssen das @ wieder schliessen, damit der Rest ganz normal funktioniert.
\end{titlepage}

\tableofcontents

\chapter{Einleitung}
\label{chap:introduction}

\enquote{Künstliche Intelligenz} ist heutzutage in jedermanns Munde. 
In den letzten Jahren hat sich die Technik rund um dieses Thema stark weiter entwickelt.
Während KI lange Zeit nur versteckt implementiert war und man sie als Nutzer oft garnicht bewusst wargenommen hat, wurde einem deren existenz aber spätestens durch die ganzen \enquote{Large Language Models} (LLMs) wie ChatGPT oder LaMDA vor Augen geführt.

Durch die sich stehts verbessernden Technologien hinter den KIs werden deren Anwenungsbereiche immer grösser.
Daher ist es nicht weit her geholt, dass KI dem Menschen gewisse Prozesse erleichtern oder abnehmen kann. 

In dieser Arbeit soll es genau um dieses Thema gehen. Wie sieht ein Berufsumfeld aus, wenn KI Technologien vermehrt Arbeiten des Menschen übernimmt und welche Chancen aber auch Risiken bringt dies mit sich.

\chapter{Wie lernt KI?}
\label{chap:ai-training}

Um alle Sachverhalte zu verstehen muss zuerst die Frage, wie eine KI erstellt, bzw. trainiert wird, geklärt werden.


\input{chap_methode.tex}

\section{Etwas mit Quellen}

Etwas mit Änderung hier am Ende.

Wenn ich eine Quelle zitieren möchte, kann ich das ganze einfach am Ende des Satzes machen \citep{example}. Oder wie \citet{example} sagt, auch mitten im Text.

\printbibliography

\end{document}
