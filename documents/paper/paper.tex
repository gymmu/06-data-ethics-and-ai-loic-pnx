\documentclass{report}

\usepackage[ngerman]{babel}
\usepackage[utf8]{inputenc}
\usepackage[T1]{fontenc}
\usepackage{hyperref}
\usepackage{graphicx}
\usepackage[german=guillemets]{csquotes}
\usepackage[a4paper]{geometry}
\usepackage{tikz}


\usepackage[
    backend=biber,
    style=apa,
    sortlocale=de_DE,
    natbib=true,
    url=false,
    doi=false,
    sortcites=true,
    sorting=nyt,
    isbn=false,
    hyperref=true,
    backref=false,
    giveninits=false,
    eprint=false]{biblatex}
\addbibresource{../references/bibliography.bib}


\title{Künstliche Intelligenz und deren Auswirkung auf die Arbeitswelt}
\author{Loïc Schneider}
\date{\today}


\begin{document}

\begin{titlepage}
    \makeatletter
    \begin{center}
        {\scshape Gymnasium Muttenz} \vspace{0.5cm}

        Informatik 2024\vspace{3.5cm}

        {\huge\bfseries \@title}

        \vspace{2cm}

        {\Large\bfseries Welche Chancen und Risiken bietet KI beim Einsatz in der Medizin?}

        \vspace{3cm}

        {\Large\itshape \@author}

        \vspace{0.5cm}

        Version vom: \@date

    \end{center}
    \makeatother

    % Ensure the image covers the entire page width and bottom margin
    \begin{tikzpicture}[remember picture,overlay]
        \node[anchor=south west,inner sep=0pt] at (current page.south west) {\includegraphics[width=\paperwidth, height=\paperheight, keepaspectratio]{TitlePicture.jpg}};
    \end{tikzpicture}
\end{titlepage}

\tableofcontents

\chapter{Einleitung}
\label{chap:introduction}

\enquote{Künstliche Intelligenz} ist heutzutage in jedermanns Munde. 
In den letzten Jahren hat sich die Technik rund um dieses Thema stark weiter entwickelt.
Während KI lange Zeit nur versteckt implementiert war und man sie als Nutzer oft garnicht bewusst wargenommen hat, wurde einem deren existenz aber spätestens durch die ganzen \enquote{Large Language Models} (LLMs) wie ChatGPT oder LaMDA vor Augen geführt.

Durch die sich stehts verbessernden Technologien hinter den KIs werden deren Anwenungsbereiche immer grösser.
Daher ist es nicht weit her geholt, dass KI dem Menschen gewisse Prozesse erleichtern oder abnehmen kann. 

In dieser Arbeit soll es genau um dieses Thema gehen. Wie sieht ein Berufsumfeld aus, wenn KI Technologien vermehrt Arbeiten des Menschen übernimmt und welche Chancen aber auch Risiken bringt dies mit sich.

\chapter{Wie lernt KI?}
\label{chap:ai-training}

Um alle Sachverhalte zu verstehen muss zuerst die Frage, wie eine KI erstellt, bzw. trainiert wird, geklärt werden.
Das Training von KIs lässt sich in mehrere Schritte unterteilen.
Besonders wichtig ist es zu verstehen, wie Aufwändig ein solches Training ist und vor allem, wie oft und viele Daten eingesetzt werden müssen.

\section{Daten sammeln}
Bevor man eine KI trainieren kann muss man zuerst einen Datensatz erstellen. Dieser Datensatz muss sehr gross sein, denn anderst als beispielsweise ein Kind braucht KI viel mehr Inputs um etwas neues zu erlernen, das Kind kann Muster schon nach wenigen Inputs erkennen und anwenden.

Diese Datensätze müssen unter beachtung verschiedener Kriterien gesammelt werden. 
Damit die KI keine falschen Dinge lernt, muss der Datensatz auf seine richtigkeit überprüft werden. Dazue gehört beispielsweise bei Texten, dass diese vereinheitlicht werden und Fehler entfernt.
Der Datensatz muss ausserdem den Einsatzbereich der KI möglichst gut abdecken bzw. representieren. Somit kann garantiert werden, dass die KI im späteren Einsatz nicht von Situationen überascht wird, die trotz des vorherigen trainings unmöglich zu lösen sind.

\section{Lernen}
Nachdem ein Datensatz vorhanden ist wird dieser Verwendet um der KI die \enquote{Intelligenz} anzueignen. 

Dazu gibt es verschiedene Lernkategorien, diese unterscheiden, wie Algorithmen Daten analyisieren und auswerten.
\subsection{überwachtes und unüberwachtes Lernen (unsupervised/supervised)}
Beim überwachten Lernen wird das Modell mit Daten trainiert, die bereits gekennzeichnet sind. Das bedeutet, dass die Daten manuell in Gruppen eingeteilt wurden, damit der Algorithmus daraus lernen kann. 
Dies eignet sich gut für Aufgaben wie Risikobewertungen, Ja/Nein-Entscheidungen oder Rechtschreibprüfungen.

\vspace{1em}
\noindent
Beim unüberwachten Lernen hingegen analysieren die Algorithmen die Daten selbstständig und erkennen Muster, ohne dass die Daten vorher in Gruppen eingeteilt wurden. 
Diese Methode wird oft genutzt, um große Datensätze zu analysieren.

\vspace{1em}
\noindent
Es gibt auch die Mischung aus den beiden Kategorien, bei denen die Daten teilweise manuell teilweise automatisch eingeteilt werden.

\subsection{ver-/bestärkendes Lernen (reinforcement)}
Beim verstärkenden Lernen lernen Algorithmen durch „Belohnungen“ und „Bestrafungen“, welche Entscheidungen gut oder schlecht sind. Dies ähnelt dem menschlichen Lernen. 
Verstärkendes Lernen wird oft bei autonomen Fahrzeugen, Robotik und Spiele-KIs eingesetzt.

\section{Überprüfen und Anpassen}
Nach dem die KI mit dem Datensatz trainiert wurde, kann sie überprüft werden. Dabei ist es wichtig einen neuen Datensatz zu verwenden, den die KI im training noch nicht gesehen hat. Somit kann ausgeschlossen werden, dass die KI lediglich die Antwort auf den Trainigsdatensatz auswendig gelernt hat und die Zusammenhänge wirklich versteht.

Bei diesem Prozess können dann Schwachstellen der KI festgestellt werden. Dadurch kann man sich nochmals auf spezifische Bereiche fokusieren und Parameter anpassen um eine stabilere und exaktere Funktion zu gewährleisten.
Es ist auch sinnvoll, sich vor Augen zuführen wieso die KI auf ein gewisses Ergebnis gekommen ist.

Dies zeigte auch ein Fall von Forschern an der Universität Türingen. \citep{tench-messy-data}
Diese haben ein neuronales Netzwerk zum indentifizieren von Bildern erstellt und haben diesem dann denn auftrag gegeben den Teil des Bildes zu zeigen, dass für die Erkennung des Objekts am relevantesten war.
Bei der erkennung von Schleien (eine Fischart) zeigte es als den relevantesten Teil stehts ein paar Finger vor einem grünen Hintergrund. Dies geschah weil der Datensatz, der beim training für die Schleien benutzt wurde, grösstenteils aus Bilder von Menschen, die den Fisch halten bestand, da dieser Fisch oft eine Art Trophäe ist.

\noindent
In diesem Fall führte ein unsauberere Datensatz also dazu, dass ein zusammenhang zwischen zwei Faktoren geschlossen wurde, die eigentlich nichts miteinander zu tun haben. Das zeigt wie wichtig es ist den Lösungsweg vom KIs zu analysieren und möglichst perfekte Datensätze bereit zu stellen.

Am ende Wird das trainierte Modell in den tatsächlichen Anwendungsbereich eingesetzt, wo es dann echte Daten verarbeitet. 
Dabei wird es jedoch noch immer überwacht um mögliche Fehler zu entdecken und das Modell durch weitere Anpassungen stetig zu verbessern.

\chapter{KI an Arbeitsplätzen}
\label{chap:ai-workingplaces}

Durch die steigenden technologischen Fähigkeiten von Maschinen durch KI, werden auch stetig neue Anwendungsbereiche geschaffen.
Auch in der Arbeitswelt nimmt KI einen grossen Einfluss. Immer mehr Arbeiten die zuvor noch von Menschen durchgeführt werden mussten, können durch Maschinen übernommen werden.
Damit verbunden steht das Risiko, dass Arbeitsplätze der automatisierung zum Opfer fallen könnten.
Trozdem gibt es viel Potential für Verbesserungen durch solche Technologien. Darunter sind zum Beispiel körperliche Entlastung oder erhöhte Sicherheitsstandarts.

Verbunden mit solchen änderungen sind viele Überlegungen. Beispielsweise müssen manche Arbeitsprofile angepasst werden, an anderen Stellen entstehrn vielleicht komplett neue.
KI gebundene System könnten unter den richtigen Umständen helfen Entscheidungen rationaler zu treffen und dadurch niemanden zu benachteiligen. 
Gleichzeitig taucht aber auch der kritische Punkt des Datenschutzes von verschiedenen Parteien wie Arbeitnehmer*innen, Empfänger*innen von Dienstleistungen oder Ähnlichem auf.

Daran, dass digitale Technologien wie KI Veränderungen in die Arbeitswelt bringen werden oder es auch schon tun, ist kein Zweifel.
Es gilt aber, sich viele Gedanken um Themen wie den Datenschutz, Entscheidungsbeschränkung oder gesetzlichen Vorlagen zu KIs zu machen.

\chapter{KI in der Medizin}
\label{chap:ai-medicine}

brbrbbrbrb

\chapter{Fazit}
\label{chap:Fazit}

Die Integration von Künstlicher Intelligenz in die Medizin bietet sowohl immense Chancen als auch erhebliche Risiken. 

KI kann die Medizin effizienter, präziser und patientenorientierter gestalten, indem sie Ärzte entlastet und durch ihre Fähigkeit die Diagnose und Behandlung von Krankheiten verbessert. 
Zudem kann KI die Forschung nach neuen Medikamenten und Behandlungsmethoden unterstützen und durch Fernüberwachung oder virtuelle Assistenten Patienten auch außerhalb von medizinischen Einrichtungen helfen.

Dennoch sind diese Vorteile nicht ohne Herausforderungen. Ethische Bedenken, Datenschutzfragen und die Zuverlässigkeit der KI-Systeme stellen bedeutende Risiken dar. 
Die Skepsis der Patienten gegenüber der Sicherheit und Funktionalität der KI kann zu einer Ablehnung von Behandlungen führen, was wiederum physische und psychische Folgen haben kann. 
Der Umgang mit sensiblen Patientendaten erfordert strikte Datenschutzmaßnahmen.
Ethische Überlegungen betreffen die Frage, wie viel Verantwortung an Maschinen abgegeben werden sollte, die letztlich auf algorithmischen Entscheidungen basieren und keine moralische Urteilsfähigkeit besitzen. 
Rechtliche Fragen zur Haftung im Falle von Fehlern durch KI-Systeme müssen geklärt werden, und es ist wichtig, dass Entscheidungen, die von KI getroffen werden, für Ärzte und Patienten nachvollziehbar sind.

\vspace{5mm} \noindent
Insgesamt muss der Einsatz von KI in der Medizin sorgfältig geplant und überwacht werden, um sicherzustellen, dass die Vorteile maximiert und die Risiken minimiert werden. 
Datenschutz, ethische Prinzipien und rechtliche Rahmenbedingungen müssen dabei stets berücksichtigt werden, um den sicheren und verantwortungsvollen Einsatz von KI in der Gesundheitsversorgung zu gewährleisten.


\nocite{*}

\printbibliography

\end{document}
