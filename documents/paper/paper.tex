\documentclass{report}

\usepackage[ngerman]{babel}
\usepackage[utf8]{inputenc}
\usepackage[T1]{fontenc}
\usepackage{hyperref}
\usepackage{csquotes}
\usepackage[a4paper]{geometry}

\usepackage[
    backend=biber,
    style=apa,
    sortlocale=de_DE,
    natbib=true,
    url=false,
    doi=false,
    sortcites=true,
    sorting=nyt,
    isbn=false,
    hyperref=true,
    backref=false,
    giveninits=false,
    eprint=false]{biblatex}
\addbibresource{../references/bibliography.bib}


\title{Künstliche Intelligenz und deren Auswirkung auf die Arbeitswelt}
\author{Loïc Schneider}
\date{\today}


\begin{document}

\begin{titlepage}
    \makeatletter % Das hier brauchen wir damit wir spezielle Befehle wie \@author verwenden können.
	\begin{center}
		{\scshape Gymnasium Muttenz} \vspace{0.5cm}

		 Informatik 2023/2024\vspace{5.5cm}

		{\huge\bfseries \@title}

		\vspace{2cm}

		{\Large\itshape \@author}

        \vspace{2cm}

        Version vom: \@date
	\end{center}
    
    \makeatother % Wir müssen das @ wieder schliessen, damit der Rest ganz normal funktioniert.
\end{titlepage}

\tableofcontents

\chapter{Einleitung}
\label{chap:introduction}

\enquote{Künstliche Intelligenz} ist heutzutage in jedermanns Munde. 
In den letzten Jahren hat sich die Technik rund um dieses Thema stark weiter entwickelt.
Während KI lange Zeit nur versteckt implementiert war und man sie als Nutzer oft garnicht bewusst wargenommen hat, wurde einem deren existenz aber spätestens durch die ganzen \enquote{Large Language Models} (LLMs) wie ChatGPT oder LaMDA vor Augen geführt.

Durch die sich stehts verbessernden Technologien hinter den KIs werden deren Anwenungsbereiche immer grösser.
Daher ist es nicht weit her geholt, dass KI dem Menschen gewisse Prozesse erleichtern oder abnehmen kann. 

In dieser Arbeit soll es genau um dieses Thema gehen. Wie sieht ein Berufsumfeld aus, wenn KI Technologien vermehrt Arbeiten des Menschen übernimmt und welche Chancen aber auch Risiken bringt dies mit sich.

\chapter{Wie lernt KI?}
\label{chap:ai-training}

Um alle Sachverhalte zu verstehen muss zuerst die Frage, wie eine KI erstellt, bzw. trainiert wird, geklärt werden.


\chapter{KI an Arbeitsplätzen}
\label{chap:ai-workingplaces}

Durch die steigenden technologischen Fähigkeiten von Maschinen durch KI, werden auch stetig neue Anwendungsbereiche geschaffen.
Auch in der Arbeitswelt nimmt KI einen grossen Einfluss. Immer mehr Arbeiten die zuvor noch von Menschen durchgeführt werden mussten, können durch Maschinen übernommen werden.
Damit verbunden steht das Risiko, dass Arbeitsplätze der automatisierung zum Opfer fallen könnten.
Trozdem gibt es viel Potential für Verbesserungen durch solche Technologien. Darunter sind zum Beispiel körperliche Entlastung oder erhöhte Sicherheitsstandarts.

Verbunden mit solchen änderungen sind viele Überlegungen. Beispielsweise müssen manche Arbeitsprofile angepasst werden, an anderen Stellen entstehrn vielleicht komplett neue.
KI gebundene System könnten unter den richtigen Umständen helfen Entscheidungen rationaler zu treffen und dadurch niemanden zu benachteiligen. 
Gleichzeitig taucht aber auch der kritische Punkt des Datenschutzes von verschiedenen Parteien wie Arbeitnehmer*innen, Empfänger*innen von Dienstleistungen oder Ähnlichem auf.

Daran, dass digitale Technologien wie KI Veränderungen in die Arbeitswelt bringen werden oder es auch schon tun, ist kein Zweifel.
Es gilt aber, sich viele Gedanken um Themen wie den Datenschutz, Entscheidungsbeschränkung oder gesetzlichen Vorlagen zu KIs zu machen.

\chapter{KI in der Medizin}
\label{chap:ai-medicine}

Ein grosses potenzielles und auch teilweise schon exisiterendes Anwendungsgebiet für Künstliche Intelligenzen liegt in der Medizin.
Dort kann diese vielseitig, beispielsweise für Diagnosen oder eine Ersteinschätzung von Patienten eingesetzt werden.

Dabei entstehen grosse Chancen wie aber auch Risiken.

\section{Chancen}
Dem vermehrten Einsatz von KI in der Bedizin stehen viele Chancen gegenüber.
Ein Problem in der Medizin ist, dass sich Ärtze zu wenig Zeit nehmen können um wirklich auf die Patienten einzugehen und eine Verbindung aufzuabauen.
Wenn KI gewisse Aufgaben übernehmen kann, würde das die Ärtzte entlasten und sie könnten sich wieder verstärkt auf die Patienten einlassen, was den Arbeitsaltag der Ärtze erleichtert so wie dem Patienten ein besseeres Gefühl zu vermitteln.


\section{Risiken}

\chapter{Quellen}
\label{chap:sources}

Nachfolgend sind die Quellen aufgelistet, teilweise sind diese auch schon in der Arbeit verlinkt.

\section{Etwas mit Quellen}

Etwas mit Änderung hier am Ende.

Wenn ich eine Quelle zitieren möchte, kann ich das ganze einfach am Ende des Satzes machen \citep{example}. Oder wie \citet{example} sagt, auch mitten im Text.

\printbibliography

\end{document}
