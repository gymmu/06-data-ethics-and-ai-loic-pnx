\chapter{KI an Arbeitsplätzen}
\label{chap:ai-workingplaces}

Durch die steigenden technologischen Fähigkeiten von Maschinen durch KI, werden auch stetig neue Anwendungsbereiche geschaffen.
Auch in der Arbeitswelt nimmt KI einen grossen Einfluss. Immer mehr Arbeiten die zuvor noch von Menschen durchgeführt werden mussten, können durch Maschinen übernommen werden.
Damit verbunden steht das Risiko, dass Arbeitsplätze der automatisierung zum Opfer fallen könnten.
Trozdem gibt es viel Potential für Verbesserungen durch solche Technologien. Darunter sind zum Beispiel körperliche Entlastung oder erhöhte Sicherheitsstandarts.

Verbunden mit solchen änderungen sind viele Überlegungen. Beispielsweise müssen manche Arbeitsprofile angepasst werden, an anderen Stellen entstehen vielleicht komplett neue.
KI gebundene System könnten unter den richtigen Umständen helfen Entscheidungen rationaler zu treffen und dadurch niemanden zu benachteiligen. 
Gleichzeitig taucht aber auch der kritische Punkt des Datenschutzes von verschiedenen Parteien wie Arbeitnehmer*innen, Empfänger*innen von Dienstleistungen oder Ähnlichem auf.

Daran, dass digitale Technologien wie KI Veränderungen in die Arbeitswelt bringen werden oder es auch schon tun, ist kein Zweifel.
Es gilt aber, sich viele Gedanken um Themen wie den Datenschutz, Entscheidungsbeschränkung oder gesetzlichen Vorlagen zu KIs zu machen.