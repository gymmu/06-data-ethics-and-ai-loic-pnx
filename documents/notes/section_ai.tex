\section{Künstliche Intelligenz}
\label{sec:ai}

In diesem Abschnitt sind meine Notizen zu künstlicher Intelligenz zu finden.

Künstliche Intelligenz ist ein Teilgebiet der Informatik und beschäftigt sich mit maschinellem Lernen \citep{ai-wikipedia}.
\newline

\subsection{Thema}
\setlength{\parindent}{0pt}
Mögliche Themen:
    \begin{itemize}
    \item Datenschutz
    \item Rechtliche Grundlagen
    \item Soziale Auswirkung von KI 
    \item Entscheidungsprozesse durch KI 
    \end{itemize}
Mögliche Fragestellungen:
    \begin{itemize}
    \item Wie kann der Datenschutz bei der Datensammlung von KI gewärleistet und transparent gemcaht werden?
    \item Welche rechtlichen Grundlagen sind Notwendig, um einen sicheren und nachhaltigen Umgang mit KI zu erschaffen?
    \item Wie verändert der Einsatz von Künstlicher Intelligenz die Arbeitswelt und welche Auswirkungen hat dies auf das psychische Wohlbefinden der Beschäftigten?
    \item Welche Chancen und Risiken bringt KI im Einsatz in der Medizin mit sich?
    \end{itemize}
\newpage

\begin{KI}
    \item Kein Konzept von richtig und falsch
    \item Nur so gut wie die Datensätze mit denen sie trainiert wurde
\end{KI}



https://www.aiweirdness.com/when-data-is-messy-20-07-03/
https://www.pcspezialist.de/blog/2021/03/17/maschinelles-lernen/

Wurde das KI-Modell nicht gegen
eine Membership Inference Attack geschützt, kann es möglich sein, mit bestimmten Eckwerten
einer Person herauszufinden, ob ihre Daten für das Training des Modells benutzt worden waren
und folglich auch, ob sie Krebs hatte. Dies geht vereinfacht gesagt so: Wurde der KI beigebracht,
dass jemand mit den Parametern genau dieser Person Krebs hat und ist eine Kombination genau
dieser Parameter höchst unwahrscheinlich, gibt die KI aber trotzdem an, dass sie sich sicher ist,
so darf davon ausgegangen werden, dass die KI die betreffende Person «wiedererkannt» hat, d.h.
sie Teil ihres Trainingsdatensatzes war und die Diagnose nicht nur eine Prognose, sondern eine
Tatsache ist. -- Rosenthal-KI