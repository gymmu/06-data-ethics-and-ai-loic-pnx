\documentclass{article}

\usepackage[ngerman]{babel}
\usepackage[utf8]{inputenc}
\usepackage[T1]{fontenc}
\usepackage{hyperref}
\usepackage{csquotes}

\usepackage[
    backend=biber,
    style=apa,
    sortlocale=de_DE,
    natbib=true,
    url=false,
    doi=false,
    sortcites=true,
    sorting=nyt,
    isbn=false,
    hyperref=true,
    backref=false,
    giveninits=false,
    eprint=false]{biblatex}
\addbibresource{../references/bibliography.bib}

\title{Notizen zum Projekt Data Ethics}
\author{Loïc Schneider}
\date{\today}

\begin{document}
\maketitle

\abstract{
    Dieses Dokument ist eine Sammlung von Notizen zu dem Projekt. Die Struktur innerhalb des
    Projektes ist gleich ausgelegt wie in der Hauptarbeit, somit kann hier einfach geschrieben
    werden, und die Teile die man verwenden möchte, kann man direkt in die Hauptdatei ziehen.
}

\tableofcontents

\section{Künstliche Intelligenz}
\label{sec:ai}

In diesem Abschnitt sind meine Notizen zu künstlicher Intelligenz zu finden.

Künstliche Intelligenz ist ein Teilgebiet der Informatik und beschäftigt sich mit maschinellem Lernen \citep{ai-wikipedia}.
\newline

\subsection{Thema}
\setlength{\parindent}{0pt}
Mögliche Themen:
    \begin{itemize}
    \item Datenschutz
    \item Rechtliche Grundlagen
    \item Soziale Auswirkung von KI 
    \item Entscheidungsprozesse durch KI 
    \end{itemize}
Mögliche Fragestellungen:
    \begin{itemize}
    \item Wie kann der Datenschutz bei der Datensammlung von KI gewärleistet und transparent gemcaht werden?
    \item Welche rechtlichen Grundlagen sind Notwendig, um einen sicheren und nachhaltigen Umgang mit KI zu erschaffen?
    \item Wie verändert der Einsatz von Künstlicher Intelligenz die Arbeitswelt und welche Auswirkungen hat dies auf das psychische Wohlbefinden der Beschäftigten?
    \item Welche Chancen und Risiken bringt KI im Einsatz in der Medizin mit sich?
    \end{itemize}
\newpage

\begin{KI}
    \item Kein Konzept von richtig und falsch
    \item Nur so gut wie die Datensätze mit denen sie trainiert wurde
\end{KI}



https://www.aiweirdness.com/when-data-is-messy-20-07-03/
https://www.pcspezialist.de/blog/2021/03/17/maschinelles-lernen/

Wurde das KI-Modell nicht gegen
eine Membership Inference Attack geschützt, kann es möglich sein, mit bestimmten Eckwerten
einer Person herauszufinden, ob ihre Daten für das Training des Modells benutzt worden waren
und folglich auch, ob sie Krebs hatte. Dies geht vereinfacht gesagt so: Wurde der KI beigebracht,
dass jemand mit den Parametern genau dieser Person Krebs hat und ist eine Kombination genau
dieser Parameter höchst unwahrscheinlich, gibt die KI aber trotzdem an, dass sie sich sicher ist,
so darf davon ausgegangen werden, dass die KI die betreffende Person «wiedererkannt» hat, d.h.
sie Teil ihres Trainingsdatensatzes war und die Diagnose nicht nur eine Prognose, sondern eine
Tatsache ist. -- Rosenthal-KI


--Ethik--
Ein weiteres Risiko sind Fehler der KI. Daher muss man sich die Frage stellen, wie viel Verantwortung man der Maschine abgibt.
Inwiefern ist es überhaupt ethisch Menschenleben von System, die durch Strom an, Strom aus funktioniern, abhängig zu machen?

Im prinzip werden Menschenleben durch KI in der Medizin auf blosse Datensätze reduziert. Wo bleibt da die Menschenwürde?
Ein moralisches bzw. ethisches Problem liegt auch darin, dass KI genau diese beiden Dinge momentan nur spärlich besitzt. 
Durch Daten lassen sich zwar rational  die besten Entscheidungen finden, allerdings ist eine komplett rationale Entscheidung teilweise nicht die optimale.

--Autonomie und Kontrolle--
Faktoren wie Datenqualität, algorithmische Fehler oder technkische Ausfälle können dazu führen, dass die KI Dinge tut, die sie nicht sollte. 
Daher macht es Sinn nicht die komplette Verantwortung an eine Maschine abzugeben, sondern immer noch eine Fachkraft als Endentscheidungsinstanz zu haben.
Sicher ist, die Entscheidung muss für Ärzte und Patienten klar ersichtlich und verständlich sein. Selbst wenn das KI-System scheinbar so gut geworden ist, dass es keine Fehler macht, 
müssen Entscheidungen noch klar verständlich sein. Bei der Einführung von digitalen Systemen besteht immer auch ein  Risiko von Cybaerangriffen.
Es kann eine ganz neue Art von Cyberkriminalität entsthen, bei der man durch geziehlte Angriffe auf Medizinische Systeme Personen im echten Leben Schaden möchte.


--Rechtliches--





--Fazit--
Die Integration von Künstlicher Intelligenz in die Medizin bietet sowohl immense Chancen als auch erhebliche Risiken. 

KI kann die Medizin effizienter, präziser und patientenorientierter gestalten, indem sie Ärzte entlastet und durch ihre Fähigkeit die Diagnose und Behandlung von Krankheiten verbessert. 
Zudem kann KI die Forschung nach neuen Medikamenten und Behandlungsmethoden unterstützen und durch Fernüberwachung oder virtuelle Assistenten Patienten auch außerhalb von medizinischen Einrichtungen helfen.

Dennoch sind diese Vorteile nicht ohne Herausforderungen. Ethische Bedenken, Datenschutzfragen und die Zuverlässigkeit der KI-Systeme stellen bedeutende Risiken dar. 
Die Skepsis der Patienten gegenüber der Sicherheit und Funktionalität der KI kann zu einer Ablehnung von Behandlungen führen, was wiederum physische und psychische Folgen haben kann. 
Der Umgang mit sensiblen Patientendaten erfordert strikte Datenschutzmaßnahmen.
Ethische Überlegungen betreffen die Frage, wie viel Verantwortung an Maschinen abgegeben werden sollte, die letztlich auf algorithmischen Entscheidungen basieren und keine moralische Urteilsfähigkeit besitzen. 
Rechtliche Fragen zur Haftung im Falle von Fehlern durch KI-Systeme müssen geklärt werden, und es ist wichtig, dass Entscheidungen, die von KI getroffen werden, für Ärzte und Patienten nachvollziehbar sind.

Insgesamt muss der Einsatz von KI in der Medizin sorgfältig geplant und überwacht werden, um sicherzustellen, dass die Vorteile maximiert und die Risiken minimiert werden. 
Datenschutz, ethische Prinzipien und rechtliche Rahmenbedingungen müssen dabei stets berücksichtigt werden, um den sicheren und verantwortungsvollen Einsatz von KI in der Gesundheitsversorgung zu gewährleisten.

\printbibliography

\end{document}
